%% start of file `my-CV-2013.tex'.
%% Copyright 2006-2013 Xavier Danaux (xdanaux@gmail.com).
%
% This work may be distributed and/or modified under the
% conditions of the LaTeX Project Public License version 1.3c,
% available at http://www.latex-project.org/lppl/.

% possible options include font size ('10pt', '11pt' and '12pt'),
% paper size ('a4paper', 'letterpaper', 'a5paper', 'legalpaper',
% 'executivepaper' and 'landscape') and font family ('sans' and 'roman')
\documentclass[11pt,a4paper,roman]{moderncv} 

% moderncv themes
% style options are 'casual' (default), 'classic', 'oldstyle' and 'banking'
\moderncvstyle{classic}

% color options 'blue' (default), 'orange', 'green',
% 'red', 'purple', 'grey' and 'black'
% \moderncvcolor{blue}

\usepackage{wangweicv}

% to set the default font; use '\sfdefault' for the default sans serif font,
% '\rmdefault' for the default roman one, or any tex font name
% \renewcommand{\familydefault}{\sfdefault}

% uncomment to suppress automatic page numbering for CVs longer than one page
%\nopagenumbers{}

% character encoding
% if you are not using xelatex ou lualatex,
% replace by the encoding you are using
\usepackage{DejaVuSansMono}
\usepackage[utf8]{inputenc}
\usepackage{fontspec,indentfirst}
\usepackage{xunicode}
\usepackage{xltxtra}


 % if you need to use CJK to typeset your resume in Chinese, Japanese or Korean
% \usepackage{CJKutf8}
\usepackage{tikz}
\usepackage{footmisc}

\XeTeXlinebreaklocale "zh"
\XeTeXlinebreakskip = 0pt plus 1pt minus 0.1pt

\newfontfamily\xingjai{"华文行楷"}
\newfontfamily\caiyun{"华文彩云"}
\newfontfamily\kai{"楷体"}
\newfontfamily\fs{"仿宋"}
\newfontfamily\li{"隶书"}
\newfontfamily\xinwei{"华文新魏"}
\newfontfamily\yao{"方正姚体"}
\newfontfamily\hei{"黑体"}
\newfontfamily\song{"新宋体"}
\newfontfamily\mshei{"微软雅黑"}

\setmainfont{"宋体"}

\renewcommand{\baselinestretch}{1.1}
\newenvironment{tightitemize}
{\begin{itemize}\setlength{\parskip}{0pt}}
{\end{itemize}}
% adjust the page margins
\usepackage[scale=0.8]{geometry}
% if you want to change the width of the column with the dates
%\setlength{\hintscolumnwidth}{3cm}
% for the 'classic' style, if you want to force the width allocated to
% your name and avoid line breaks. be careful though, the length is normally
% calculated to avoid any overlap with your personal info;
% use this at your own typographical risks...
%\setlength{\makecvtitlenamewidth}{10cm}

\makeatletter
\tikzset{
    tl@startyear/.append style={
        xshift=(0.5-\tl@startfraction)*\hintscolumnwidth,
        anchor=base
    }
}
\makeatother

% quote, optional, remove / comment the line if not wanted.
% \myquote{MAKE A LITTLE SPACE, MAKE A BETTER PLACE.}{Michael Jackson: <<Heal the world>>}

% to show numerical labels in the bibliography (default is to show no labels);
% only useful if you make citations in your resume
%\makeatletter
%\renewcommand*{\bibliographyitemlabel}{\@biblabel{\arabic{enumiv}}}
%\makeatother

% CONSIDER REPLACING THE ABOVE BY THIS
%\renewcommand*{\bibliographyitemlabel}{[\arabic{enumiv}]}

% bibliography with mutiple entries
%\usepackage{multibib}
%\newcites{book,misc}{{Books},{Others}}
%-------------------------------------------------------------------------------
%                                 content
%-------------------------------------------------------------------------------
\begin{document}
% \begin{CJK*}{UTF8}{gbsn} % to typeset your resume in Chinese using CJK
\makecvtitle

\footnotetext[1]{项目优先级:~Low, ~Normal, ~High.}
% \footnotetext[2]{个人简历采用~\LaTeX~编写.}
% \footnotetext[2]{在项目中担任的角色.}

\section{\li{基本信息}}
\cvdoubleitem{姓~名:}{王维}{政治面貌:}{中共党员}
\cvdoubleitem{性~别:}{男}{籍~贯:}{湖北武汉}
\cvdoubleitem{民~族:}{汉族}{个人主页:}{\texttt{\httplink{www.cnblogs.com/ict-wangwei}}}

\section{\li{教育经历}}

% arguments 3 to 6 can be left empty
% \cventry{2007--2011}{Degree}{Institution}{City}{\textit{Grade}}{Description}
% \tlcventry{2007--2011}{Degree}{Institution}{City}{\textit{Grade}}{Description}
\tllabelcventry{2007}{2011}{2007-2011}{\hei{学士}}{计算机科学与技术}{华中科技大学}{武汉}{}

\tlcventry{2011}{0}{\hei{硕士在读}}{网络数据科学与技术重点实验室}{中国科学院计算技术研究所}{北京}{}

\section{\li{学士毕业论文}}
\cvitem{论文题目}{\emph{基于非确定性错误的录制重放调试工具ReBranch的完善}}
\cvitem{指导老师}{蒋文斌}

\section{\li{专业技能}}
\cvdoubleitem{操作系统}{Ubuntu, Fedora, CentOS}{数据库}{MySQL, MongoDB, Memcached}
\cvdoubleitem{编程语言}{C/C++, Python, Java, Lua, Bash}{文档编写}{\LaTeX, Office软件等}
\cvdoubleitem{日常工具}{Git, GDB, Vim, CMake/Make, Screen}{外语水平}{英语~6~级, 自学日语}

\section{\li{硕士期间参与项目}}
\tllabelcventry{2011}{2012}{2012.3-2012.6}{High\footnotemark[1]}{基于libvirt的虚拟机集群管理工具}{编程语言:C\&Python}{}{
\begin{tightitemize}
		\item 关键技术:~libvirt, winpcap, epoll.
		\item 基本介绍:为某网络安全中心写的一套虚拟机集群管理工具, 能批量管理全国各地节点的虚拟机集群, 监控虚拟机的cpu,内存,网络运行状况并接受反馈.
\end{tightitemize}}
\tllabelcventry{2011}{2012}{2012.7-2012.9}{Normal}{实验室p2p项目的虚拟机集群环境搭建和项目部署}{编程语言:C++\&Python}{}{
\begin{tightitemize}
		\item 关键技术:~libvirt, kvm, samba, BT.
		\item 基本介绍:为某网络安全中心的p2p相关项目搭建kvm集群虚拟机集群平台, 部署并维护p2p相关程序, 效果获得该中心一致好评.
\end{tightitemize}}
\tllabelcventry{2012}{2013}{2012.10-2013.1}{High}{实验室网络监控项目数据后台处理}{编程语言:C++\&Python}{}{
\begin{tightitemize}
		\item 关键技术:~Thrift, nlpbamboo, stanfordner, curl, regex, mysql, mongo.
		\item 基本介绍:类似于人立方的系统, 对各种信息通道(论坛, 新闻, 微博, 学术等)抓取的数据进行命名实体识别, 关系属性抽取等处理, 最后进行人物信息归档.
\end{tightitemize}}
\tllabelcventry{2013}{0}{2013.2}{High}{基于论坛的社区发现}{编程语言:C++}{}{
\begin{tightitemize}
		\item 关键技术:~CFinder, metis, mysql, mongo.
		\item 基本介绍:通过对论坛的归档信息进行分析, 找出意见领袖, 互动频繁, 稳定共现等关系的人物或者群体. 
\end{tightitemize}}

\section{\li{业余兴趣与开源项目}}
\tllabelcventry{2012}{0}{2012.7-2012.10}{muduo-re}{\textsc{C++}}{https://github.com/sofartogo/muduo-re}{C++ 基于Reactor模式的网络编程库的代码重构}{}
\tllabelcventry{2012}{0}{2012.9}{OSSC}{\textsc{C\&Shell}}{https://github.com/sofartogo/OSSC}{阿里云开放存储服务(Open Storage Service: OSS)~C~SDK, 首届阿里云开发者大赛获奖作品(最佳实用奖: 奖金~5~万人民币整)}{}
\tllabelcventry{2012}{0}{2012.10}{oss-cpp-sdk}{\textsc{C++\&Shell}}{https://github.com/sofartogo/oss-cpp-sdk}{阿里云开放存储服务(Open Storage Service: OSS)~C++~SDK}{}
\tllabelcventry{2012}{0}{2012.10}{lsm-tree-re}{\textsc{C}}{https://github.com/sofartogo/lsm-tree-re}{基于跳表的lsm-tree的代码重构}{}
\tllabelcventry{2012}{0}{2012.12}{websql}{\textsc{C}}{https://github.com/sofartogo/websql}{结构化数据操作的rest-api开放服务}{}
\tllabelcventry{2013}{0}{2013.03}{google-styleguide-chinese-translation}{\textsc{reStructuredText}}{https://github.com/sofartogo/google-styleguide-chinese-tranlation}{google-styleguide 各种编程语言风格指南的中文翻译}{}

\section{\li{本科及硕士阶段所获主要奖项}}
\cvitem{2008.11}{\kai{国家励志奖学金}}
\cvitem{2009.11}{\kai{国家励志奖学金}}
\cvitem{2010.03}{\kai{北美数模竞赛二等奖}}
\cvitem{2010.04}{\kai{软件设计师(中级)}}
\cvitem{2012.11}{\kai{首届阿里云开发者大赛最佳实用奖}}
\cvitem{2012.12}{\kai{中科院计算技术研究所天玑团队优秀学生奖}}

\section{\li{兴趣爱好}}
\cvlistdoubleitem{篮球}{爬山}
\cvlistdoubleitem{弹吉他}{算24点}
% if you are typesetting your resume in Chinese using CJK;
% the \clearpage is required for fancyhdr to work correctly with CJK,
% though it kills the page numbering by making \lastpage undefined
\clearpage
% \end{CJK*}
\end{document}

%% end of file `my-CV-2013.tex'.
